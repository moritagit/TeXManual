%第3章
\documentclass[class=jreport, crop=false, preview=false, dvipdfmx, fleqn]{standalone}
\usepackage{../../preamble/preamble_TeXManual}
\begin{document}
\chapter{環境}

先の章までで平の文書は打てるようになりました。
この章では,「環境」を使って箇条書きや中央揃え,右寄せなどをしていきます。



\section{環境}
まず,環境(environment)というものについて説明しようと思います。
環境とは,\verb|\begin{...}|と\verb|\end{...}|で対になった命令のことを言います。
\verb|\begin{○○}|...\verb|\end{○○}|となっている環境のことを○○環境と言います。
環境の内章は外章とは異なることが多く,例えばフォントや文字サイズを環境内で変えても,
その環境の外のフォント・文字サイズは変わりません。

では,以下でいくつかの例を見ていきましょう。



\section{quote環境と文字寄せ}
quote環境は,文章を引用するときに使うためのものです。
文頭にスペースが入ってそれっぽくなります。

\begin{IOTeX}
普通の部分だよ~
\begin{quote}
ここはquote環境の中。\\
\tiny ここで色々変えても
\end{quote}
元に戻る~
\end{IOTeX}


また,文字の位置を変え,中央揃えや右寄せにすることもできます。

\begin{center}
\begin{tabular}{ll}
flushleft環境 & 左寄せ \\
flushright環境 & 右寄せ \\
center環境 & 中央揃え
\end{tabular}
\end{center}

なお,1行なら,これらはそれぞれ,
\verb|\raggedright|,\verb|\centering|,\verb|\raggedleft|
というコマンドでも実現できます。



\section{箇条書き}
また別の環境の例として箇条書きを挙げたいと思います。


\subsection{itemize環境}
普通に「\textbullet」などで始まる箇条書きです。

\begin{IOTeX}
箇条書きといっても
\begin{itemize}
\item 記号
\item 番号
\item 見出し
\end{itemize}
といったパターンがあります。
\end{IOTeX}


また,itemize環境を入れ子にすると,
記号が変わっていきます。

\begin{IOTeX}
\begin{itemize}
\item 1段階目
  \begin{itemize}
  \item 2段階目
    \begin{itemize}
    \item 3段階目
      \begin{itemize}
      \item 4段階目
      \end{itemize}
    \end{itemize}
  \end{itemize}
\end{itemize}
\end{IOTeX}

なお,頭の記号を変えることもできます。

この頭の記号を出力する命令は,
第1~4段階目についてそれぞれ,
\begin{ITeX}
\labelitemi, \labelitemii, \labelitemiii, \labelitemiv
\end{ITeX}
です。
これを定義し直せば記号を変えることができます。
例えば,1段階目の記号{\textbullet}(\verb|\textbullet|)を和文の「・」(中黒)に変えるときは,
\begin{ITeX}
\renewcommand{\labelitemi}{・}
\end{ITeX}
をプリアンブル(\verb|\documentclass|と\verb|\begin{document}|の間)に書けばOKです。

また,一部の記号だけ変えたいときは,
その部分の\verb|\item|命令にオプションをつけて,
\begin{ITeX}
\item[☆]
\end{ITeX}
などのようにしてください。


\subsection{enumerate環境}
これは番号で始まる箇条書きです。

\begin{IOTeX}
箇条書きといっても
\begin{enumerate}
\item 記号
\item 番号
\item 見出し
\end{enumerate}
といったパターンがあります。
\end{IOTeX}

また,enumerate環境を入れ子にすると,
数字の種類が変わっていきます。

\begin{IOTeX}
\begin{enumerate}
\item 1段階目
  \begin{enumerate}
  \item 2段階目
       \begin{enumerate}
       \item 3段階目
           \begin{enumerate}
           \item 4段階目
           \end{enumerate}
        \end{enumerate}
  \end{enumerate}
\end{enumerate}
\end{IOTeX}

こちらでも頭の番号を変えることを考えます。
第1~4段階目を出力する命令はそれぞれ,
\begin{ITeX}
\labelenumi, \labelenumii, \labelenumiii, \labelenumiv
\end{ITeX}
であるので,これらを定義直せばOKです。
具体的に事例を見ていきましょう。
第1段階の番号を変えることを考えます。
その他の段階については,iをii, iii, ivに変えてください。

まず,数字の後のピリオドを取るには,
\begin{ITeX}
\renewcommand{\labelitemi}{\theenumi}
\end{ITeX}
また,数字に()をつけるには,
\begin{ITeX}
\renewcommand{\labelitemi}{(\theenumi)}
\end{ITeX}
ローマ数字(小文字)(i., ii., iii., iv.)にするには,
\begin{ITeX}
\renewcommand{\theenumi}{\roman{enumi}}
\end{ITeX}
をそれぞれプリアンブルに書いてください。

ローマ数字(小文字)の他にも,
算用数字(デフォルト),英小文字,英大文字,ローマ数字(大文字)といったものを使うこともできます。
そのためにはそれぞれ
\verb|\arabic, \alph, \Alph, \Roman|を使えばよいです。

また,参考ですが,enumitemパッケージを読み込めば,
この番号の設定を簡単にすることができます。
例えば,
\begin{ITeX}
\begin{enumerate}[label=例 \arabic*]
\end{ITeX}
とすれば,番号の付き方が「例 \ 1,例 \ 2,例 \ 3...」となります
(空白も反映されます)。
このオプション部分の\verb|\arabic*|は,他の\verb|\roman*|などにすることもできます。
その際,*を忘れないようにしてください。 

他の例ですが,例えば数字を丸で囲んで箇条書きにしたいときは,
\begin{IOTeX*}
\begin{enumerate}[label = \textcircled{\scriptsize \theenumi}]
\item あ
\item い
\item う
\end{enumerate}
\end{IOTeX*}
のようにしてください。
なお,\verb|\textcircled{\scriptsize 数字}|で,
とりあえず簡単ですが丸囲い数字を表現できます。
emathの\verb|\maru|やotfパッケージの\verb|\ajMaru|などもっとキレイな丸はいっぱいあるので,
拘る人は是非調べてみて下さい。


\end{document}