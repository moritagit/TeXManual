\documentclass[class=jreport, crop=false, preview=false, dvipdfmx, a4paper, 14Q, fleqn]{standalone}
\usepackage{../../preamble/preamble_TeXManual}
\begin{document}

\chapter{その他の話題}

まだ触れられていない話題の中で僕が知っているものについて,
とりあえず列挙と簡単な紹介だけしておこうと思います。
今後書き足すかもしれません。

{
\setlength{\tabcolsep}{3pt}
\renewcommand{\arraystretch}{1.5}
\begin{longtable}{clcp{27zw}}
\textbullet & 索引 & : &
文書の終わりに索引を付けることができます。\\
\textbullet & 引用・参考文献 & : &
引用マークや参考文献を簡単に付けることができます。
論文やレポートを書くときには必須ですね。\\
\textbullet & 欧文フォント & : &
欧文フォント(アルファベット等)についての話です。\\
\textbullet & 縦書き & : &
{\TeX}で縦書きをすることもできます。
小説等を書きたい人などには必須ですね。
tarticleやtbookといったドキュメントクラスを使えばすぐにできます。\\
\textbullet & マクロの作り方 & : &
コマンドや環境を自分で作ることができます。
xparseパッケージを使うとより高度なものが作れると思います。\\
\textbullet & listings & : &
listingsパッケージについて紹介します。
様々なプログラミング言語(含\TeX)のソースコードを貼るときに非常に便利です。
情報系のレポートや論文を出すときには必須ですね。\\
\textbullet & \TikZ & : &
{\TeX}で図を描きます。
簡単な線画ならすぐ描けるようになります。
回路図なんかも描けるので非常に便利です。
公式ドキュメントがかなり充実しているので,
描きたいものをドキュメントから探し,
そのあたりの説明を読むことで大体のものは描けます。
このマニュアルのタイトルのページもこの{\TikZ}で作りました。
図が必要になったときは是非試してみて下さい。 \\
\textbullet & tcolorbox & : &
表において色付きやセルの結合をもっと自由に行いたい場合,
tcolorboxパッケージが便利です。
listingsをもっとデザインしたいときにも使えます。
例えば,このマニュアルの目次はこのtcolorboxを使って作っています。
日本語のサイトもちらほらありますが,
これも公式ドキュメントが充実しているので,
そちらを使ってもいいかもしれません。\\
\textbullet & スタイルファイル(.sty)の作り方 & : &
スタイルファイル,つまりパッケージを自分で作ることができます。
意外と簡単にできるので,よく使うマクロが増えてきたら作ってみるのもいいと思います。\\
\textbullet & クラスファイル(.cls)の作り方 & : &
クラスファイル(jsarticleなど)を自分で作ることができます。
ページレイアウトなどが決まったときはクラスファイルにしてしまってもよいかもしれません。\\
\textbullet & 化学式 & : &
化学式,化学反応式や複雑な構造式といったものも書くことができます。
これにはmhchemパッケージとchemfigパッケージを使います。
「\href{http://doratex.hatenablog.jp/entry/20131203/1386068127}{TeXによる化学組版}」
と
「\href{http://doratex.hatenablog.jp/entry/20141212/1418393703}{chemfigパッケージによる構造式描画}」
の2つのサイトを読めばかなり多くの化学式を打てるようになります。
また,公式ドキュメントも充実しているので,
是非見てみて下さい。\\
\textbullet & オンラインで{\TeX}をする & : &
{Cloud\TeX}やOverLeafなどを使えばオンラインでも{\TeX}ができます。\\
\textbullet & Python\TeX & : &
Ruby,Pythonなどのプログラミング言語と{\TeX}を組み合わせることができます。
具体的には,{\TeX}中にコードを埋め込んで,
それを実行することができます。
{\TeX}に用意されているマクロだけでは作れないものも
作ることができる可能性を秘めた面白いパッケージです。
\end{longtable}
}


\end{document}