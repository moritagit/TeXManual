\documentclass[class=jreport, crop=false, preview=false, dvipdfmx, fleqn]{standalone}
\usepackage{../../preamble/preamble_TeXManual}
\begin{document}
\chapter{文書を書き始める前に}
\label{ch:preparation}

この章では文章を書き始める前の準備について書きたいと思います。


\section{インストール}
\label{install}
インターネット経由でインストールする場合は,
「TeX インストール」とでも検索すればいい感じのサイトが上の方に出てきます。
そのサイトに従ってやってください(僕もこれでやりました)。
もし奥村晴彦さんの「{\LaTeXe}美文書作成入門」を購入していれば,
付属のCDからできるのでその場合はそちらでやってください
(それ持ってたらこれじゃなくてそっち読んだ方がいいと思います)。



\section{使ってみよう}
\TeX のインストールが終わったら,さっそく簡単な文書を作ってみようと思います。
定番のあれを出力してみましょう。
前節のどちらかの方法で{\TeX}をインストールした場合,
TeXworksというデスクトップアプリがインストールされていると思います。
これを開いて,以下の表の左側,入力の方を打ってみて下さい。
もちろん半角でお願いします。
なお,\verb*| |は半角スペースを表しています。

\begin{IOtcb}
\begin{verbatim*}
\documentclass{jsarticle}

\begin{document}

Hello, \TeX !
\[ \int dx = x + C \]

\end{document}
\end{verbatim*}
\tcblower
Hello, \TeX !
\[ \int dx = x + C \]
\end{IOtcb}

入力しただけでは何も起こりませんね。
まだ実行していないからです。
実行するには,TeXworksの左上あたりにある「
\tikz{\fill [color=green] (10pt,0) circle [radius=5pt];
		\fill [color=red] (13pt,0pt)--(8.5pt,2.598076pt)--(8.5pt,-2.598076pt)--cycle;}		%ボタンの図
\hspace{-4pt}」
のボタン(タイプセットボタン)を押してください。
上の表の右側のような出力がされたでしょうか?
されなかったときは,スペルミスなどを見直してみて下さい。
ちゃんと出力されたら文書作成成功です。
あとはこれがちょっと複雑になるだけなので,
あまり身構えなくても大丈夫です。

次章でこの例の入力で何故この出力になるのか説明します。



\section{ファイルの保存}
{\TeX}ファイルを保存するときはファイル名の最後に .texをつけてください
(拡張子を.texにしてください)。
テキストファイル(~.txt)で作成した文書を{\TeX}のファイルにするときは,
「名前を付けて保存」をし,
ファイル指定を「すべてのファイル」にしてから
 .texをつけて保存してください。
また,ファイル名はアルファベット・数字・アンダーバー({\_})のみでつけるのがよいでしょう。


以上でとりあえず日本語や英語のみの文章を打てるようになりました。
次章から実際に文章を書いていきましょう。


\end{document}