%第x章
\chapter{正しくきれいな文書の書き方}
\label{ch:right-words}

この章では,数式や日本語,英語の
正しい書き方やきれいな書き方を紹介していきます。
物理,数学寄りが多いです。


\section{出版物の常識}
出版物には色々な常識,テンプレがあります。
主に文字サイズや空白などです。
その辺を軽く触れておこうと思います。

まず,文字サイズは「14Q」というのが一般的です。
これはdocumentclassのオプションに「14Q」とつければそうなるので,
細かいところまで拘る人は是非。

また,「四分空き」というものがあります。
これは,半角文字と全角文字の間に,
全角の$1/4$くらいの幅のスペースを入れるというものです。
この設定は,\verb|\xkanjiskip|をいじって設定します。
\begin{ITeX}
\xkanjiskip 0.25zw plus 0zw minus 0zw
\end{ITeX}
とすればいいと思います。
ただ,これでは文尾がバラバラになってしまう可能性があるので,
\begin{ITeX}
\xkanjiskip 0.25zw plus 0.125zw minus 0.125zw
\end{ITeX}
などとしても良いと思います。

また,細かいのですが,日本語の文章における
句読点についても触れておこうと思います。
句点は縦書き横書き共通で「。」なのですが,
読点は縦書きでは「,」,横書きでは「,(全角コンマ)」とするのが標準となります。
細かいところまで気にする人は是非気を付けてみて下さい



\section{数式中の文字の字体}
\label{sec:char-in-math}

物理の話になってしまいますが,
数式の中で,
物理量と非物理量の字体について扱います。
ここを間違っている人は多いので,是非目を通してみて下さい。


\subsection{物理量}
物理量とは何かというと,
単位を持っていて(無次元数もありますが...),
何かしらの式で表される,
物理的な量です。
具体的に例を挙げれば,
長さ$x$とか力$F$,圧力$P$などのようなものです。

この物理量は,
数式中ではイタリック体(斜体)で書くのが標準です。
$x$は自然ですが,$\mathrm{x}$は違和感ありますよね。
{\TeX}では数式中に入れたアルファベットは自動でイタリック体になるので,
これに注意して書く,ということはありません。


\subsection{非物理量}
では何に注意するかというと,
非物理量です。
これは,具体的に言えば物体Aの「A」などです。
これは別に物理的な量ではないので,
イタリック体にするのは間違いです。

数式中でもこれらの非物理量はローマン体(立体,普通の形)で書きます。
例えば,終端速度$v_{\mathrm{f}}$の「f」はfinalのfなのでローマン体ですし,
理論熱効率の$\eta_{\mathrm{th}}$の「th」もtheoryからきているので,これもローマン体です。
添え字の多くはその変数を特徴づけるためのものであり,
非物理量であることが多いです。
ですので,添え字をつけるときは,
\verb|\mathrm{}|というコマンドでローマン体に直すのに気をつけましょう。

なお,中には定圧比熱$C_p$,定積比熱$C_V$のように
物理量が添え字になっているものもあるので,
添え字は全部ローマン体だ,というわけではありません。

また,定数を表す文字(ex.ばね定数$k$,プランク定数$h$など)についてですが,
これは変化しない物理量を表しているので,
イタリック体で大丈夫です。

ここで,勘のいい人は,
座標軸を表す$x, y, z$や,微分を表す$d$は物理量ではないから,
ローマン体で書かなければならないのか,と思うかもしれません。
これらに関しては,
もちろん非物理量ですのでローマン体で書くのが正しいのですが,
慣習的にイタリックで書いてしまっているので,
それでも大丈夫です。
文書内で統一されていればOKなので,
お好きな方を使ってください。



\section{単位}
長さなどの単位には,
国際標準化機構により標準が定められています。
ここでは,それに則った単位の書き方について紹介します。

前節の内容から,
単位は物理量を表してはいないので,
ローマン体で書くのが正しいです。
割と多くの人がイタリック体で書いてしまっているので,
是非とも注意してください。

文字式と数値式で書き方が異なるので,
別々に紹介していきます。


\subsection{文字式の単位}
文字式の単位については,
\begin{itemize}
\item 式との間にスペースを入れない
\item 〔〕(亀甲括弧)でくくる
\item ローマン体で書く
\end{itemize}
の3つの約束を守ればよいです。


\subsection{数値につける単位}
数値につける単位は,
\begin{itemize}
\item 式との間に半角スペースをいれる
\item 括弧はつけない
\item ローマン体で書く
\end{itemize}
の3つの約束を守って書いてください。
数値に括弧付きで単位を付けている人は少なくないので,
注意してください。
例えば,わかりづらいかもしれませんが,\\
\hspace{5zw} \SI{1.0}{m}, \ \ \ \SI{1.0}{kg} \\
のようにする,ということです。

ただ,角度を弧度法で表すときの「${}^{\circ}$」との間にはスペースは不要です(例:\ang{60})。
また,{\%}や分など,単位ではなく記号とみなされるものとの間にもスペースは不要です。
そのまま$60\text{分}$や$25{\%}$などと書いてください。


\subsection{siunitxパッケージを使う}
上記のように,単位の付け方には様々な制約があります。
これを全部守って書こうとすると,
$(2.0 \times 10^{-3} \ \mathrm{m}) \times (5.0 \times 10^{3} \ \mathrm{m/s})$と打つだけで,\\
\hspace{3zw} \verb*|(2.0 \times 10^{-3} \ \mathrm{m}) \times (5.0 \times 10^{3} \ \mathrm{m/s})| \\
のようになって,
式が長くなり見通しが悪くなります。
また,括弧の数も多く,ここに分数を入れようものなら
まずエラーするでしょう。
これを簡単にするために,
{\TeX}にはsiunitxというパッケージが用意されています。
使い方は,
\begin{ITeX}
\SI{数値}{単位}
\end{ITeX}
です。
これだけで,数値の後に適切なスペースが入り,
単位がローマン体で出力されます。
これなら式が不用意に長くなりすぎず,
エラーも起こしにくいでしょう。

また,数値の部分は,電卓のような書き方をすることもできます。
具体的に例を見ると,
\begin{ITeX}
\SI{2.0e-3}{m}
\end{ITeX}
とすれば,\SI{2.0e-3}{m}のように,
eの後の数字が10の指数となって現れます。
なお,この数字に括弧をつけるとエラーするので気を付けてください。

また,単位だけを出力したければ,
\begin{ITeX}
\si{単位}
\end{ITeX}
というコマンドも用意されています。

なお,\verb|\SI|コマンドの2つ目の引数(単位の方)に
記号系の単位({\%}や分)を入れてもスペースは入ってしまうので,
そこは自分で打ってください。

このsiunitxは,他にも様々なことができますが,
ここではとりあえず\verb|\SI|と\verb|\si|の紹介にとどめておきます。
他の便利なコマンドについては,
各種パッケージについての部を見てください。



\section{文章中の数式}
文章中(インライン)の数式についてです。
といっても,
背の低い数式(分数を含まないなど)はそのままきれいに出力されるので,
あまり問題はありません。
問題となるのは,分数や積分記号を含む,
背の高い数式です。
これらを文章中に入れると,
通常は小さくなって見づらくなります。
そこで,普通の大きさで表示するために\verb|\displaystyle|等を用いると,
今度はその上下の行に干渉して全体のバランスが悪くなります。

インラインに分数を入れる場合は,「/」で分数を表して,\\
\hspace{5zw} 「コインを投げて裏が出る確率は1/2であるが...」 \\
などとするのが一般的です。

ただ,これでは複雑な数式は表現しづらいし読みづらくもあると思う人もいるかもしれません。
その場合は\verb|\dfrac|や\verb|\cfrac|などを使っていただければ,
大きく分数が表示されて読みやすくなると思います。
ただ,そのままでは上下の行と近づきすぎて読みづらくなるので,
\verb|\lineskiplimit|と\verb|\lineskip|を設定するのがよいと思います。
そちらについては第\ref{ch:pagelayout}章\ref{sec:lineskip}節の
行送りについての部分を読んでください。

