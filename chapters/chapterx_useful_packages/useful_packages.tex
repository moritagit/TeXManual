%第x章
%随時追加
\chapter{各種パッケージの使い方について}
\label{ch:packages}

ここまで,fancthdrやhyperref,importなど便利なパッケージを紹介してきました
ここでは,まだ紹介していない便利なパッケージについて説明したいと思います。



\section{パッケージのインストールとファイルの置き場所}
\label{sec:install-packages}

{\TeX}には色々なパッケージがありますが,
中には最初にダウンロードした中に入っていないものもあります。
こういったものは,どこかからダウンロードしてくるしかありません。
まずは「パッケージ名 ダウンロード」とでも検索して,
ダウンロードしてください。

ダウンロードしたzipファイルの展開をすると,
欲しいパッケージのファイル,
.styファイル(スタイルファイル)が手に入ります。
これが,パッケージの本体となります。

しかし,ただダウンロードと展開をしただけでは,
{\TeX}がこの.styファイルを見つけることができず,
パッケージが見つからないといったエラーが出ます。
よって,この.styファイルを,
{\TeX}が見つけられるところに置く必要があるのです。

この,「{\TeX}が見つけられるところ」には色々ありますが,
一番簡単なのは,そのパッケージを使う.texファイルと同じフォルダに入れることです。

ただこれでは何度も使うときに面倒だし美しくないです。
{\TeX}が探してくれるディレクトリに.styファイルを置けば
万事解決です。
そのディレクトリは,
windowsでは,w32texディレクトリ内の,
「C:/w32tex/share/texmf-distl/tex/latex」
です。
なお,p{\LaTeX}専用のものは,最後のディレクトリを「platex」にしてください。
ここに適当なディレクトリを作って,
その中に.styファイルを置いていけばよいでしょう。

また,texmf-local直下に「ls-R」ファイルがあるときは,
コマンドプロンプトでmktexlsrを実行してください。

この,パッケージが置いてあるディレクトリは,
例えばamsmathなら,
コマンドプロンプトで\\
\hspace{5zw} kpsewhich amsmath.sty \\
を実行すればわかります。
頑張って探してみて下さい。



\section{siunitx}
単位をつけるのに便利な\verb|\SI|コマンドが入っているsiunitxパッケージですが,
他にもいろいろ便利なコマンドがあるので紹介しておきます。


\subsection{単位自体について}
まずは単位についてのお話です。
単位は,\verb|\metre|や\verb|\gram|といったコマンドでも出力できます。
また,「${}^{-1}$」を表す「/」は\verb|\per|で出すことができます。
また,\si{\square\metre}は,\verb|m^2|でも出せますし,
\verb|\square\metre|や\verb|\metre\squared|ともできます
(3乗は\verb|\cubic|でできます)。
単位のコマンドについてはドキュメントにすべて載っているので是非見てみて下さい。
なお,僕には見た目では\verb|\metre|とm等の違いは判りませんでしたので,
好きな方を使えばよいと思います。
「${}^{-1}$」を表す「/」も「/」をそのまま打って大丈夫です。

また,角度についてですが,
弧度法で表すとき,通常では
\begin{ITeX}
60^{\circ}
\end{ITeX}
のように打たなければなりません。
これは面倒ですが,
siunitxには便利なコマンドが用意されています。
\begin{ITeX}
\ang{60}
\end{ITeX}
と打てば\ang{60}と出力されます。
もし分や秒までいれたいなら,;で区切って
\begin{ITeX}
\ang{60;2;5}
\end{ITeX}
とすれば\ang{60;2;5}と出力されます。


\subsection{単位の区切り}
単位の区切りは,乗算なら「.」,除算は「\verb|\per|(/)」で行います。
例えば,次のような感じです。
\begin{ITeX}
\SI{1.0}{kg.m/s^2}
\end{ITeX}
(または\verb|\SI{1.0}{\kilo\gram.\metre\per\square\second}|)
これは\SI{1.0}{kg.m/s^2}のように出力されます。

この乗算の区切りですが,
デフォルトでは半角スペースよりちょっと狭いくらいのスペースです。
これを「$\cdot$(\verb|\cdot|)」にしたい人もいると思います。
そのとき,\verb|\SI{1.0}{kg{\cdot}m/s^2}|などとしてもよいですが,
これではスペースがかなり空いてしまう上面倒です。

それを解決するオプションが,「inter-unit-product」です。
\begin{ITeX}
\si[inter-unit-product = \ensuremath{\cdot}]{kg.m}
\end{ITeX}
とすれば,\si[inter-unit-product = \ensuremath{\cdot}]{kg.m}と出力されます。
これでもまだスペースが広いような気がするので,
僕は\verb|\cdot|の前後に\verb|\hspace{-1.5pt}|を入れて使っています。

ただ,これだといちいちそれぞれのコマンドにオプションを付けねばならず,
面倒極まりないのではと思う人もいるかも知れません。
siunitxのコマンドには様々なオプションを付けることができますが,
それを一括設定するためのコマンドが用意されています。
それが,\verb|\sisetup|です。
プリアンブルに,引数にオプションを入れた\verb|\siunitx|コマンドを打つことで,
文書全体でのsiunitxのコマンドにそのオプションが適用されます。
hyperrefパッケージなどにも同様のコマンドがありますね。
プリアンブルに,
\begin{ITeX}
\sisetup{inter-unit-product = \ensuremath{\cdot}}
\end{ITeX}
と書けば,
文書中全てのsiunitxで付けた単位の区切りが「$\cdot$」になります。

また,除算の区切りについても触れておきます。
除算は\verb|\per|で区切りますが,
デフォルトでは\verb|\per|を付けた後の1つだけに「${}^{-1}$」が付きます。
例えば次のようになります。\\
\hspace{5zw} \verb|\si{J \per mol \per K}|
$\rightarrow$ \si[per-mode = reciprocal]{J \per mol \per K}

これを「/」に変更したいときは,オプションに
\begin{ITeX}
per-mode = symbol
\end{ITeX}
とします。\\
\hspace{5zw} \verb|\si[per-mode = symbol]{J \per mol \per K}|
$\rightarrow$ \si[per-mode = symbol]{J \per mol \per K}

なお,デフォルトのper-modeは「reciprocal」というものなので,
一部だけデフォルトのものに戻したいときはそのコマンドのオプションに
「per-mode = reciprocal」と書いてください。


\subsection{数値の演算}

掛け算を出力したいとき,
通常ではいちいち\verb|\times|を打たなければなりませんが,
siunitx中のコマンド\verb|\num|を使えば,
\begin{ITeX}
\num{1.2 x 3.2 x 6.4}
\end{ITeX}
と打つだけで\num{1.2 x 3.2 x 6.4}と出力されます。

この「x」を「*」に変えたいときは,
\begin{ITeX}
input-product = *
\end{ITeX}
というオプションを付けます。
こちらの方が好きな人は是非設定してみて下さい。

また,分数についてですが,
\begin{ITeX}
quotient-mode = fraction
\end{ITeX}
というオプションを追加することで,
\hspace{5zw} \verb|\num{1/3}| $\rightarrow$ $\num[fraction-function = \dfrac]{1/3}$ \\
というように打つことができるようになります。
\verb|\num{1.62/2e-4}|のような打ち方も可能です。

これは\verb|\frac|の形式で分数を出力しますが,
これを例えば\verb|\dfrac|に変えたければ,
\begin{ITeX}
fraction-function = \dfrac
\end{ITeX}
というオプションを追加すればよいです。

なお,\verb|\num|コマンドの引数の中に「()」は入れられないので
(誤差を表す特殊な役割があるため),
()を入れたいときはその中で\verb|\num|コマンドを使う形になると思います。


\subsection{その他}
他にも,
自動で四捨五入したり,
数字の配列を作ったり,
単位の色を変えたり,
新しい単位を定義をしたり,
単位付きの数値の表をきれいに出したり,
太字にしたりフォントを変えたりできます。
また,細かい設定は\verb|\sisetup|でオプションを指定すれば変えることができます。
その辺は是非ドキュメントを読んでやってみて下さい。



%\section{physics}




\section{文字を回す}
文字通り文字を回します。
Qiitaを読んでいたらこのパッケージを作った方の記事を見つけて,
面白かったのでまとめちゃいました。
まずはtcfaspinパッケージをダウンロードし,
プリアンブルで読み込んでください。

これの使い方は,\\
\hspace{5zw} \verb|\faSpin{文字}| \\
のようにするだけです。
これをAdobe ReaderなどのAdobe製品で開けばOKです。
こうするだけで,\\
\vspace{3zw} \\
\hspace{5zw} \faSpin{回ってる?} \\
\vspace{3zw} \\
のように文字が回ります。

また,これは入れ子にすることもでき,\\
\vspace{3zw} \\
\hspace{5zw} \faSpin{回れ\faSpin{回れ\faSpin{回れ\faSpin{回れ}}}} \\
\vspace{3zw} \\
のようにもできます。

インライン形式(\$ に挟まれた形)であれば数式も回せますよ \\
\vspace{3zw} \\
\hspace{5zw} \faSpin{$\Gamma (z) = \int_0^{\infty} t^{z-1} e^{-t} dt$} \\
\vspace{3zw} \\

これを有効にするには,
2回実行しなければならないので注意してください。

なお,この部分は,
「\href{http://qiita.com/zr_tex8r/items/2dbaabff6a795661d413}{これからの時代はLaTeXを回転させよう!}」
から引用しました。

