\chapter{その他の話題}

まだ触れられていない話題の中で僕が知っているものについて,
とりあえず列挙と簡単な紹介だけしておこうと思います。
今後書き足すかもしれません。

{
\setlength{\tabcolsep}{3pt}
\renewcommand{\arraystretch}{1.5}
\begin{tabular}{clcp{27zw}}
\textbullet & 索引 & : &
文書の終わりに索引を付けることができます。\\
\textbullet & 引用・参考文献 & : &
引用マークや参考文献を簡単に付けることができます。
論文やレポートを書くときには必須ですね。\\
\textbullet & 欧文フォント & : &
欧文フォント(アルファベット等)についての話です。\\
\textbullet & 縦書き & : &
{\TeX}で縦書きをすることもできます。
小説等を書きたい人などには必須ですね。\\
\textbullet & マクロの作り方 & : &
コマンドや環境を自分で作ることができます。
xparseパッケージを使うとより高度なものが作れると思います。\\
\textbullet & listings & : &
listingsパッケージについて紹介します。
様々なプログラミング言語(含\TeX)のソースコードを貼るときに非常に便利です。
情報系のレポートや論文を出すときには必須ですね。\\
\textbullet & \TikZ & : &
{\TeX}で図を描きます。
簡単な線画ならすぐ描けるようになります。
回路図なんかも描けるので非常に便利です。\\
\textbullet & tcolorbox & : &
表において色付きやセルの結合をもっと自由に行いたい場合,
tcolorboxパッケージが便利です。\\
\textbullet & スタイルファイル(.sty)の作り方 & : &
スタイルファイル,つまりパッケージを自分で作ることができます。
意外と簡単にできるので,よく使うマクロが増えてきたら作ってみるのもいいと思います。\\
\textbullet & クラスファイル(.cls)の作り方 & : &
クラスファイル(jsarticleなど)を自分で作ることができます。
ページレイアウトなどが決まったときはクラスファイルにしてしまってもよいかもしれません。\\
\textbullet & オンラインで{\TeX}をする & : &
{Cloud\TeX}などを使えば,オンラインでも{\TeX}ができます。\\
\textbullet & Python\TeX & : &
Ruby,Pythonなどのプログラミング言語と{\TeX}を組み合わせることができます。
具体的には,{\TeX}中にコードを埋め込んで,
それを実行することができます。
{\TeX}に用意されているマクロだけでは作れないものも
作ることができる可能性を秘めた面白いパッケージです。
\end{tabular}
}

