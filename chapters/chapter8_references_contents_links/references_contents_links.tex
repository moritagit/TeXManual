%第7章
\chapter{相互参照・目次・リンク}
\label{ch:references-contents-link}

この章では,相互参照と目次,
そしてリンクを扱います。
これらは文書内を検索する際に非常に便利ですので,
是非使えるようになっておきましょう。



\section{相互参照}
\label{sec:references}

\verb|\label{ラベル}|としてラベルを付けた
章や節,図や表,数式などは,
\verb|\ref{ラベル}| コマンドを使って参照することができます。
また,そのページ数を参照したいときは,
\verb|\pageref{ラベル}|というコマンドを使います。
ただ,2回実行しないと正しく数字が現れないことに注意してください。
このラベル名は,章はch:,節はsec:,図はfig:,表はtab:,数式はeq:で
始める,というように統一しておくとよいです。
それぞれへのラベルの付け方は,
今まで扱った通りですので,
そちらをご覧ください。

2回実行しないといけないのは,
1回目の実行でlabelの情報を別のファイルに書き込み,
2回目の実行でrefにlabelの情報を与えていくからです。
これは次節で扱う目次でも同じです。
{\TeX}には2回実行しないと正しく表示されないものが
しばしばあるので,2回実行する癖をつけてもいいかもしれません。

ここで注意なのですが,
間違って同じ名前のラベルを2個設定してしまうと,
1回目の実行ではエラーを吐かないのですが,
2回目の実行からエラーしてしまいます。
なお,僕の環境では「!LaTeX Error: Missing \verb|\begin{document}|」
と出ました。
これを解決するには,
ラベルの情報を記録している.tocファイルを削除するしかありません。
これに気付くには,エラーの前に出る
「LaTeX Warning」で,同じラベルが2個あるといった主旨のものが
ありますので,
これを見つけたら.tocファイルを削除してください。



\section{目次}
目次を出力するには,
\begin{ITeX}
\tableofcontents
\end{ITeX}
とするだけでOKです。
ただ,2回実行しないとちゃんと出力されないので
注意してください。

他にも,\verb|\listoffigures|,\verb|\listoftables|で
図目次,表目次が出力されます。
第\ref{ch:pictures}章第\ref{subsec:pictures-auto-distribution}節でも触れましたが,
\verb|\caption{説明}|の説明がそのまま目次となります。
これをもっと短めにしたければ,
オプションで指定してください。

目次にどこまで細かく見出しを出力するかは簡単に変更できます。
節(\verb|\section|)までにするには,
\begin{ITeX}
\setcounter{tocdepth}{1}
\end{ITeX}
小節(\verb|\subsection|)までなら
\begin{ITeX}
\setcounter{tocdepth}{2}
\end{ITeX}
とすればよいです。

また,「*」つきの,番号が付かない見出しについて
(\verb|\section*{はじめに}|など)は,
番号だけでなく目次にも出力されません。
これを目次に出力したかったら,
\begin{ITeX}
\section*{はじめに}
\addcontentsline{toc}{section}{はじめに}
\end{ITeX}
または
\begin{ITeX}
\section*{はじめに}|
\addcontentsline{toc}{section}{\numberline{}はじめに}
\end{ITeX}
のようにします。
後者は,他の節等で番号が入っている分だけ字下げします。

なお,ドキュメントクラスがbookのときは,
序文や後書きは\verb|\frontmatter|,\verb|\backmatter|を使う方がよいです。



\section{リンク}
リンクについて説明します。
リンクを貼るには,hyperrefパッケージを使います。
オプションにドライバ名を指定できますが,
\verb|\documentclass|にdvipdfmxオプションを付けておけば十分でしょう。
また,日本語を使うには,hyperrefの後で
pxjahyperパッケージを読み込む必要があります。
まずはプリアンブルに
\begin{ITeX}
\usepackage{hyperref}
\usepackage{pxjahyper}
\end{ITeX}
と書きましょう。


\subsection{PDF内リンク}
hyperref(とpxjahyper)を読み込めば,
目次や相互参照,また次章で扱う参考文献は
自動的にリンクが有効になり,
クリックすれば参照場所に飛べるようになります。

また,この他にも,好きな文字にリンクを貼ることができます。
そのためにまず,リンクで飛ぶ先の文字を以下のようにしてください。
\begin{ITeX}
\hypertarget{リンク名}{リンク先文字}
\end{ITeX}
次に,リンクを有効にしたい文字を次のようにしてください。
\begin{ITeX}
\hyperlink{リンク名}{リンク元文字}
\end{ITeX}
このようにすれば,リンク元文字をクリックするだけで
リンク先文字に飛ぶことができます。


\subsection{WEBページへのリンク}
WEBページへのリンクとは
つまりURLのことですが,
これにリンクをつけるには
\begin{ITeX}
\url{URL}
\href{URL}{テキスト}
\end{ITeX}
のようにします。
URLには{\TeX}では特殊文字として扱われるものが
入っていることが多いですが(ex. \%,\~,\_など),
そのまま入れて大丈夫です。

{\TeX}の実行結果としてPDFを開いているときは
リンクが使えないことがありますが,
最初からPDFとして開けば基本的には
リンクは有効になっています。

例えば,この章は \\
「\url{https://texwiki.texjp.org/?hyperref}」\\
と \\
「\href{http://www.isc.meiji.ac.jp/~mizutani/tex/link_slide/hyperlink.html}{ハイパーリンク付きLaTeX文書}」\\
というサイトを参考にしました。


\subsection{その他}
他にも,メールアドレスや外部ファイルをリンク先にすることができます。
メールアドレスを指定するときは,
\begin{ITeX}
\href{mailto:アドレス}{テキスト}
\end{ITeX}
としてください
(テキストにはアドレスを使うのがよいと思います)。

外部ファイルを参照するには,
\begin{ITeX}
\href{ファイル名}{テキスト}
\end{ITeX}
とします。
ここで,このファイル名はパスで指定します。
パスについては第\ref{ch:file-management}章まで。

また,画像には
\verb|\hyperimage{画像のURL}|,
としてリンクを貼ることができます。


\subsection{hyperrefのオプションとPDFの文書情報}
hyperrefを使えば楽にリンクが貼れることがわかりました。
ただ,デフォルトだと枠で囲まれる形になるので,
これが嫌ならオプションを付けて変更します。
また,同時にPDFに文書情報をつけることもできますので,
そちらも一緒に扱います。

以下に示したもの以外にも多くのオプションがありますが,
使いそうなものだけ抜粋したので
もし不足していれば
「\href{https://texwiki.texjp.org/?hyperref}{TeX Wiki hyperref}」や
hyperrefのドキュメント等を見てください。

\begin{table}[H]
\label{tab:options-of-hyperef}
\begin{center}
\begin{longtable}{lp{35zw}}
colorlinks &
	リンクを枠囲いでなく文字色の変化で表す。
	デフォルトはfalse。
	colorlinks=trueとして変更する。\\
hidelinks &
	リンクの色や枠を無効にする。
	デフォルトはfalse。\\
anchorcolor &
	anchorテキストの色。
	デフォルトはblack。\\
linkcolor &
	PDF内参照用のリンクの色。
	デフォルトはred。\\
linkbordercolor &
	PDF内参照用のリンクの枠囲いの色。
	デフォルトはred。\\
urlcolor &
	URL参照用リンクの色。
	デフォルトはcyan。\\
urlbordercolor &
	URL参照用リンクの枠囲いの色。
	デフォルトはcyan。\\
citecolor &
	参考文献参照用のリンクの色。
	デフォルトはgreen。\\
citebordercolor &
	参考文献用リンクの枠囲いの色。
	デフォルトはgreen。\\
filecolor &
	外部ファイル参照用のリンクの色。
	デフォルトはcyan。\\
filebordercolor &
	外部ファイル参照用リンクの枠囲いの色。
	デフォルトは[rgb]\{0, .5, .5\}。\\
bookmarks &	
	PDFにしおりをつける。
	デフォルトはfalse。
	変更はcolorlinksのときと同様。\\
bookmarksnumbered &
	しおりに章番号等をつける。
	デフォルトはfalse。\\
pdftitle &
	PDFのタイトルを設定する。
	デフォルトでは何も設定されていない(空)。\\
pdfauthor &
	PDFの作者を設定する。
	デフォルトでは空。\\
pdfkeywords &
	PDFのキーワードを設定する。
	デフォルトでは空。
\end{longtable}
\end{center}
\end{table}

以上が使いそうなオプションとなります。
このオプションは本来,
\begin{ITeX}
\usepackage{hyperref}|
\end{ITeX}
のオプションに設定するものですが,
長くなりすぎる場合があるので別にコマンドが用意されています。
\verb|\hypersetup{}|の引数にオプションを入れていけば大丈夫です。
例えばこのマニュアルの設定は以下のようになっています。

\begin{ITeX}
%文書情報とリンク
\hypersetup{%
    bookmarks=true,%
    bookmarksnumbered=true,%
    hidelinks,%
    colorlinks=true,%
    linkcolor=black,%
    urlcolor=cyan,%
    citecolor=black,%
    filecolor=magenta,%
    setpagesize=false,%
    pdftitle={LaTeX2e入門マニュアル},%
    pdfauthor={R. Morita},%
    pdfkeywords={TeX; LaTeX}
    }
\end{ITeX}

もしリンクを貼りたくないラベルがあれば,
\verb|\ref*{label}|のように
\verb|\ref|コマンドに「*」をつけてください。

なお,このhyperrefは曲者らしく,
使うと色々な弊害があるらしいです。
今のところ特に実害はないので気にしていませんが,
ちょっと凝ったことをしようとしている人はご注意を。
ネットでは,graphicxの後に読み込むとエラーしたなどという記事もあるので,
困ったらその辺を見てみて下さい。

