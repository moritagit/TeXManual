%参考文献

\begin{thebibliography}{99}
\bibitem{奥村:美文書}
	奥村晴彦/黒木裕介.
	改訂第7版{\LaTeXe}美文書作成入門.
	技術評論社, 2017
\bibitem{TeXwiki}
	{\TeX} Wiki.
	\href{https://texwiki.texjp.org/}{https://texwiki.texjp.org/}.
\bibitem{shiga}
	Kumazawa Yoshiki.
	\href{http://www.biwako.shiga-u.ac.jp/sensei/kumazawa/tex/}{http://www.biwako.shiga-u.ac.jp/sensei/kumazawa/tex/}.
\bibitem{qiita:ルビ}
	zr{\_}tex8r.
	LaTeX文書で“美しい日本の”ルビを使う ~pxrubricaパッケージ~.
	\href{http://qiita.com/zr_tex8r/items/42466cbcbeb670a3a2dc}{\url{http://qiita.com/zr_tex8r/items/42466cbcbeb670a3a2dc}}, 2017.
\bibitem{hatena:tcolorbox}
	あざらし.
	tcolorboxの基本-物理とTeXに関する話題.
	\href{http://texmedicine.hatenadiary.jp/entry/2015/12/17/000339}{http://texmedicine.hatenadiary.jp/entry/2015/12/17/000339}, 2015.
\bibitem{水谷:hyperlink}
	水谷正大.
	ハイパーリンク付きLaTeX文書.
	\href{http://www.isc.meiji.ac.jp/~mizutani/tex/link_slide/hyperlink.html}{\url{http://www.isc.meiji.ac.jp/~mizutani/tex/link_slide/hyperlink.html}}, 2014
\bibitem{zr:pxchfon}
	PXchfonパッケージ~pLaTeX文書のフォントを簡単に変更~.
	\href{http://zrbabbler.sp.land.to/pxchfon.html}{http://zrbabbler.sp.land.to/pxchfon.html}, 2016.
\bibitem{天地有情}
	天地有情.
	[LaTeX] textpos --- テキストブロックを任意の場所に配置.
	\href{http://konoyonohana.blog.fc2.com/blog-entry-218.html}{http://konoyonohana.blog.fc2.com/blog-entry-218.html}, 2016.
\bibitem{nikki:pdfpages}
	にっき:pdfpages.
	\href{http://abenori.blogspot.jp/2015/07/pdfpages.html}{http://abenori.blogspot.jp/2015/07/pdfpages.html}, 2015.
\bibitem{sankichi:subfiles}
	sankichi92.
	分割したLaTeXファイルをsubfilesを使ってコンパイルする.
	\href{http://qiita.com/sankichi92/items/1e113fcf6cc045eb64f7}{http://qiita.com/sankichi92/items/1e113fcf6cc045eb64f7}, 2017.
\bibitem{ShareLaTeX:multi-file}
	Multi-file LaTeX projects.
	\href{https://ja.sharelatex.com/learn/Multi-file_LaTeX_projects}{\url{https://ja.sharelatex.com/learn/Multi-file_LaTeX_projects}}, 2017.
\bibitem{doraTeX:mhchem}
	doraTeX.
	TeXによる化学組版.
	\href{http://doratex.hatenablog.jp/entry/20131203/1386068127}{http://doratex.hatenablog.jp/entry/20131203/1386068127}, 2013.
\bibitem{doraTeX:chemfig}
	doraTeX.
	chemfigパッケージによる構造式描画.
	\href{http://doratex.hatenablog.jp/entry/20141212/1418393703}{http://doratex.hatenablog.jp/entry/20141212/1418393703}, 2014.
\bibitem{qiita:rotatechar}
	zr{\_}tex8r.
	これからの時代はLaTeXを回転させよう!.
	\href{http://qiita.com/zr_tex8r/items/2dbaabff6a795661d413}{\url{http://qiita.com/zr_tex8r/items/2dbaabff6a795661d413}}, 2017
\bibitem{fontbear}
	FONT BEAR.
	\href{https://fontbear.net/}{https://fontbear.net/}.
\end{thebibliography}

