%タイトル
\title{\LaTeXe 入門}
\author{R. Morita}


%パッケージ
\usepackage{layout}					%ページレイアウト出力
\usepackage{lscape}					%横置き(紙は縦)
%\usepackage{pdflscape}				%横置き(紙も横置き)
\usepackage{tcfaspin}					%文字を回すのだ!
\usepackage{pythontex}				%PythonTeX
\usepackage[deluxe]{otf}				%フォント
\usepackage[noalphabet]{pxchfon}


%フォント
\setminchofont{ipam.ttf}
\setboldminchofont{GenEiGothicM-Bold}
\setmediumgothicfont{GenEiGothicM-SemiBold}				%ゴシック = 源暎
\setboldgothicfont{GenEiGothicM-Bold}
\setxboldgothicfont{GenEiGothicM-Heavy}


%tcolorboxのlibraryとsetup
\tcbuselibrary{skins, listings, documentation}
\tcbset{%
	%keywords bold = false,%			%謎エラー(未定義?)
	color command = colcom,%
	color environment = colenv,%
	}


%hyperrefのsetup
\hypersetup{%
	pdftitle = {LaTeX2e入門マニュアル},%
	pdfauthor = {R. Morita},%
	pdfkeywords = {TeX; LaTeX}
	}


%ページレイアウト
%jsarticleの初期値に設定
\setlength{\oddsidemargin}{0pt}
\setlength{\topmargin}{4pt}
\setlength{\headheight}{20pt}
\setlength{\textheight}{634pt}
\setlength{\textwidth}{453pt}
\setlength{\marginparsep}{18pt}
\setlength{\marginparwidth}{18pt}
%ヘッダー周り
\setlength{\headsep}{60pt}
\setlength{\voffset}{-40pt}


%ヘッダー&フッター
\fancypagestyle{normal}{%
	%ヘッダー
	\lhead{\leftmark}
	\rhead{\rightmark}
	\renewcommand{\headrule}{}
	%フッター
	\cfoot{\thepage}
	}
\pagestyle{normal}


%色
\definecolor{lightcyan}{rgb}{.875,1,1}
\definecolor{lightyellow}{rgb}{1,1,.9375}
\definecolor{colcom}{RGB}{0, 0, 255}
\definecolor{colenv}{RGB}{0, 100, 0}


%TikZ
\newcommand{\TikZ}{%
	Ti{\textit{k}}Z%
	}


%入出力ディスプレイ
\newtcblisting{IOTeX}{%
	skin = bicolor,%
	drop shadow,%
	listing side text,%
	colframe = blue!5!black,%
	colback = lightcyan,%
	colbacklower = lightyellow,%
	fontlower = {\setlength{\mathindent}{1zw}},%
	after = {\noindent},%
	listing options = {style=tcblatex, texcsstyle=*\color{blue}},%
	}

\newtcblisting{IOTeX*}{%
	skin = bicolor,%
	drop shadow,%
	colframe = blue!5!black,%
	colback = lightcyan,%
	colbacklower = lightyellow,%
	fontlower = {\setlength{\mathindent}{1zw}},%
	after = {\noindent},%
	listing options = {style=tcblatex, texcsstyle=*\color{blue}},%
	}

\newtcolorbox{IOtcb}{%
	skin = bicolor,%
	drop shadow,%
	sidebyside,%
	colframe = blue!5!black,%
	colback = lightcyan,%
	colbacklower = lightyellow,%
	after = {\noindent},%
	fontlower = {\setlength{\mathindent}{1zw}},%
	}


%別行立ての入力表示
\newtcblisting{ITeX}{%
	boxrule = 0pt,%
	size = minimal,%
	left = 5zw,%
	colback = lightcyan,%
	listing options = {style=tcblatex, texcsstyle=*\color{blue}},%
	listing only
	}



%以下不要

%入出力ディスプレイ(環境、verbatimは打ってね)
\newcommand{\IOchange}{%
	\vspace{3pt}
	\end{minipage} &
	\begin{minipage}[c]{0.4\hsize}
	}
\newenvironment{IOdisplay}{%
	\begin{center}
	\arrayrulewidth=2pt
	\arrayrulecolor{white}
	\begin{tabular}{>{\columncolor{lightcyan}}l|>{\columncolor{lightyellow}}l}
	\multicolumn{1}{>{\columncolor{cyan}}l|}{\bf {\textcolor[gray]{1}{入力}}} 
	&
	\multicolumn{1}{>{\columncolor{orange}}l}{\bf {\textcolor[gray]{1}{出力}}}
	\tabularnewline
	\begin{minipage}[c]{0.4\hsize}
	\vspace{6pt}
	}{%
	\end{minipage} \tabularnewline
	\end{tabular}
	\end{center}
	\vspace{\baselineskip}
	}





%マニュアルの書き方のルール
%	自分の言葉で書く
%	1行目に「%第n章」と入れる
%	2行目に \chapter{題名} を入れ、1行あけて章の概要説明
%	節間は3行あける
%	小節間は2行あける

