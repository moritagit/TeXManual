\documentclass[class=jreport, crop=false, preview=false, dvipdfmx, a4paper, 14Q, fleqn]{standalone}
\usepackage{../../preamble/preamble_TeXManual}

\begin{document}
\chapter*{はじめに}


\section*{本マニュアルについて}
これは,{\TeX}(テフorテック)を初めてやる人,また,ちょっとやったことがあるけど軽い数式しか打てない人など,初心者~中級者くらいを対象として書かれた{\LaTeXe}(ラテフorラテック)のざっくりしたマニュアルです。
細かいとこは飛ばしてとりあえずすぐ使えるようになりたいという人を想定して書いているので,ざっくりしています。
また,細かいことは僕もわかっていないことが多いのであしからず。


\section*{{\TeX}について}
ざっくり説明すると,{\TeX}はキレイな文書をPDFで作るためのソフトです。
文書を作成するならwordでいいじゃんと思いますが,
{\TeX}はwordよりも細かいところまでキレイに書けますし(特に数式),
オープンソースなのでお金の色々を気にしなくて済んだりします。
読み方は前述のとおりテフかテックで,
普通のアルファベットで書くときはTeX(LaTeX2e)と書きます。

{\TeX}は,普通のテキストエディタ(メモ帳)とプログラミングのハイブリッドみたいなものです。
目的は文書作成(テキストエディタ)なのですが,書き方がプログラミングっぽいってだけです。
普通の数式を打つだけなら大して難しくないので,プログラミングと聞いてびびらないでくださいね。\\

以上で前書きを終わりたいと思います。
目次の後から具体的な書き方について説明していきます。
細かいところは端折ったりもしてるので,わからないところがあったら普通にググってくださいね~。

\end{document}